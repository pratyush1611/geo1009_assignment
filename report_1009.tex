\documentclass[a4paper]{article}

%%%%%%%% CREATE DOCUMENT STRUCTURE %%%%%%%%
%% Language and font encodings
\usepackage[english]{babel}
\usepackage[utf8x]{inputenc}
\usepackage[T1]{fontenc}
%\usepackage{subfig}

%% Sets page size and margins
\usepackage[a4paper,top=3cm,bottom=2cm,left=2cm,right=2cm,marginparwidth=1.75cm]{geometry}

%% Useful packages
\usepackage{amsmath}
\usepackage{graphicx}
\usepackage[colorlinks=true, allcolors=blue]{hyperref}
\usepackage{caption}
\usepackage{subcaption}
\usepackage{sectsty}
\usepackage{float}
\usepackage{titling} 
\usepackage{blindtext}
\usepackage{xcolor}
\usepackage{apacite}

\definecolor{darkgreen}{rgb}{0.0, 0.4, 0.0}
\graphicspath{ {images/} }
%%%%%%%% DOCUMENT %%%%%%%%
\begin{document}
%%%% Title Page
\title{\textbf{SDI and open data: A case study on India} }

\author{\small \textbf{Camera Ford, Pratyush Kumar} \\
        \small Delft University of Technology, Faculty of Architecture and The Built Environment \\
        \small The Netherlands\\
        \small Email: \tt{c.a.ford@student.tudelft.nl, p.kumar-12@student.tudelft.nl}
}


\date{}
\maketitle

\begin{abstract}
\noindent Insert Abstract here Insert Abstract here Insert Abstract here Insert Abstract here Insert Abstract here 
Insert Abstract here Insert Abstract here Insert Abstract here Insert Abstract here Insert Abstract here Insert Abstract here Insert Abstract here Insert Abstract here Insert Abstract here Insert Abstract here Insert Abstract here Insert Abstract here Insert Abstract here 
\end{abstract}

\small \textbf{Keywords: } SDI; NSDI; SSDI

%%%% SECTIONS
%% Introduction
\section{Introduction}
\begin{itemize}
    \item sdi definition 
    \item sdi and india breif
    \item how we explore in the paper
\end{itemize}
% para below can be expanded furthur
Geospatial technologies have been in use for quite some time, be it for the purpose of mapping or for decision making based on inputs obtained from different sensors. At a regional scale, these data-sets, when obtained from various agencies, producers and users of data,  become difficult to manage without a specified standard on what will be the contents of each of these datasets. To tackle this problem, SDI or Spatial Data Infrastructure have been implemented. Coined by the US National Research Council in 1993 as \textit{ "a framework of technologies, policies, and institutional arrangements that together facilitate the creation, exchange, and use of geospatial data and related information resources across an information-sharing community"} \cite{council1993toward}, SDI enables organisations and concerned authorities to make collaboration and sharing of information more streamlined and standardised.




\newpage
\bibliographystyle{apacite}
\bibliography{ref.bib}

\end{document}